\begin{center}
	\Huge
	30\,MHz vágású aluláteresztő szűrő RH-SDR-hez
	
	\Large
	HA5KFU projekt
	
	\Large
	Keresztes Botond
\end{center}


\section*{Specifikáció}

Cél egy olyan $50\,\Omega$ be-, és kimenetű $30\,MHz$ vágású aluláteresztő szűrő tervezése, mely $44\,MHz$ frekvencián már legalább $40\,dB$-es elnyomást biztosít. Ez azért szükséges, hogy a későbbi keverésből adódó tükörfrekvenciákat megbízhatóan el lehessen nyomni. Továbbá követelmény, hogy a szűrő legfeljebb hetedfokú legyen, ez a főmérnök (Kiss Ádám) által szabott határ.


\section*{Első iteráció}

A szűrő méretezéséhez az \url{https://rf-tools.com/lc-filter/} linken található programot használtam, itt specifikálni lehet a szűrő típusát (Csebisev, Butterworth stb.), jellegét (alul-, felüláteresztő stb.), vágási frekvenciá(ka)t, illetve ahol releváns, egyéb paramétereket (pl. áteresztő sávi hullámosság). Ezen kívül kiszámíttathatók az elemek pontos értékei, vagy a szabványos sorokban található értékekből felépített közelítése.

Az első próbálkozás (\ref{fig:csebisev_390n}-es ábra) egy hetedfokú Csebisev szűrő volt, ugyanis ez ígérkezett megfelelő meredekséget adni a határok megtartásához. Ezt teljesítette is, bár csak úgy, ha $0,3\,dB$-nyi hullámosságot engedélyeztem a zárósávban. Ez a struktúra két különböző kondenzátor, illetve induktivitás értéket ($360\,nH$, $390\,nH$) tartalmazott. A számok relatív közelsége miatt adta magát, hogy áttérjek egyfajta tekercsértékre (\ref{fig:csebisev_360n}-es ábra), így csökkentve a különböző alkatrészek mennyiségét.

Az \texttt{rf-tools} tervezőben sajnos nincs lehetőség egyes értékek manuális átírására, így a további szimulációkat LTSpice-ban végeztem\footnotemark. Ennek az az előnye, hogy exportálhatók a szimulációs eredmények. Ezzel a lehetőséggel éltem is, és Octave segítségével ábrázoltam az eredményeket (lásd: Függelék).
\footnotetext{A forrás és terhelés ellenállás osztása miatt itt a kimenet alapból $-6\,dB$ csillapítással számítódik, ezt az ábrázolásnál ki kell (lehet) kompenzálni.}

Az tekercsek egységes $360\,nH$-re cserélése ugyan kicsit kijjebb tolja a $-3\,dB$-es pontot, viszont növeli az áteresztő sávi hullámosságot, és csökkenti az elnyomás mértékét $44\,MHz$-en. Mérlegelendő, hogy az alkatrészek számán szeretnénk-e spórolni, vagy inkább egy jobb karakterisztikájú szűrőre van szükségünk.

Erre a diskurzusra viszont sor sem kerülhetett, ugyanis ezen szűrők úgy lettek méretezve, hogy az \texttt{rf-tools} az E24-es sorban található szabványos értékeket használta fel a tervezéshez, viszont hamar kiderült, hogy ezen alkatrészek elérhetősége (névelegesen a $360\,nH$-s tekercsé) igencsak csekély.

\newpage


\section*{Második iteráció}

Világossá vált, hogy ha bármiféle választékot szeretnénk az alkatrészek beszerzésekor, akkor át kell térni E12-es értéksorra. Ebben természetesen \enquote{szellősebben} találhatók elemértékek, viszont az elérhetőségük jóval jobb.

Ádámmal konzultálva arra jutottunk, hogy a prioritás a $44\,MHz$-es elnyomási kritérium teljesítése, és hogy ez akár kierőszakolható a törésponti frekvenciának a csökkentésével. A határértékekkel való kis játék után adódott, hogy a sávszélesség $29,5\,MHz$-re választásával elérhető a specifikált elnyomás, mindössze $0,2\,dB$ hullámossággal -- az \texttt{rf-tools} ez számította, a szimuláció (\ref{fig:csebisev_295}-as ábra) ezzel nem teljesen egyezett.

Ez egy észszerű kompromisszumnak mutatkozik az eredeti követelmények és az alkatrész elérhetőség között, így ezt a változatot szándékozzuk használni.


\section*{Alkatrészek}

Alkatrészek keresésekor fő szempont volt az ár és az elérhetőség és természetesen a minőség. Ez a tekercs esetén magas rezonanciafrekvenciát és alacsony ellenállást jelent, kondenzátoroknál pedig igyekeztem olyan márkából is beválogatni egyet-egyet, melyekről mondták nekem, hogy azok jó minőségűek nagyobb frekvenciákon is. Jelenleg úgy alakult, hogy ez minden esetben a Kemet, de ilyen még pl. a Vishay vagy a Murata is.

Kiválasztásra került több lehetséges érték, ha esetleg gondok merülnének fel a rendeléssel (értsd: elfogy a raktárkészlet), illetve a kondenzátoroknál két különböző méretből is listáztam terméket, ugyanis itt nem elhanyagolható az árbeli különbség. Az általam javasolt alkatrészek minden táblázatban és a hivatkozások között is vastagon ki vannak emelve, de alkalmazásuk akár vita tárgyát képezheti.

\subsection*{Tekercs}

\begin{center}
	\begin{tabular}{ c c c c c c c c }
		\textbf{Forgalmazó} & \textbf{Gyártó} & \textbf{Érték} & \textbf{Méret} & \textbf{Ellenállás} & \textbf{SRF} & \textbf{Ár [db]} & \textbf{Link}\\
		
		\hline \\
		
		\textbf{Farnell} & \textbf{TDK} & \textbf{390\,nH} & \textbf{1210} & \textbf{0,45\,$\Omega$} & \textbf{250\,MHz} & \textbf{kb. 60\,Ft} & \textbf{\cite{tdk_tekercs}} \\[10pt]
		
		Farnell & WE & 390\,nH & 0603 & 2,3\,$\Omega$ & 350\,MHz & kb. 30\,Ft & \cite{we_tekercs} \\[10pt]
		
		Lomex & KOA & 390\,nH & 1210 & 0,45\,$\Omega$ & 250\,MHz & kb. 45\,Ft & \cite{koa_tekercs}
		
		
	 	\end{tabular}
\end{center}


\subsection*{Kondenzátor}

\begin{center}
	\begin{tabular}{ c c c c c c c c }
		\textbf{Forgalmazó} & \textbf{Gyártó} & \textbf{Érték} & \textbf{Méret} & \textbf{Tűrés} & \textbf{Egyéb} & \textbf{Ár [db]} & \textbf{Link}\\
		
		\hline \\
		
		\textbf{Farnell} & \textbf{Kemet} & \textbf{150\,pF} & \textbf{0805} & \textbf{5\%} & \textbf{NP0} & k\textbf{b. 20\,Ft} & \textbf{\cite{kemet_150p_0805}} \\[10pt]
		
		Farnell & WE & 150\,pF & 0805 & 10\% & X7R & kb. 20\,Ft & \cite{we_150p_0805} \\[10pt]
		
		Farnell & Walsin & 150\,pF & 0603 & 5\% & NP0 & kb. 6\,Ft & \cite{walsin_150p_0603} \\[10pt]
		
		Farnell & Kemet & 150\,pF & 0603 & 5\% & NP0 & kb. 9\,Ft & \cite{kemet_150p_0603} \\[10pt]
		
		\hline \\
		
		\textbf{Farnell} & \textbf{Kemet} & \textbf{270\,pF} & \textbf{0805} & \textbf{5\%} & \textbf{NP0} & \textbf{kb. 21\,Ft} & \textbf{\cite{kemet_270p_0805}} \\[10pt]
		
		Farnell & Multicomp & 270\,pF & 0805 & 5\% & NP0 & kb. 23\,Ft & \cite{multicomp_270p_0805} \\[10pt]
		
		Farnell & Walsin & 270\,pF & 0603 & 10\% & X7R & kb. 7\,Ft & \cite{walsin_270p_0603} \\[10pt]
		
		Farnell & Kemet & 270\,pF & 0603 & 5\% & NP0 & kb. 15\,Ft & \cite{kemet_270p_0603} \\[10pt]
		
	\end{tabular}
\end{center}